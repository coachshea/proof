\documentclass[12pt,]{article}
\usepackage{lmodern}
\usepackage{amssymb,amsmath}
\usepackage{ifxetex,ifluatex}
\usepackage{fixltx2e} % provides \textsubscript
\ifnum 0\ifxetex 1\fi\ifluatex 1\fi=0 % if pdftex
  \usepackage[T1]{fontenc}
  \usepackage[utf8]{inputenc}
\else % if luatex or xelatex
  \ifxetex
    \usepackage{mathspec}
  \else
    \usepackage{fontspec}
  \fi
  \defaultfontfeatures{Ligatures=TeX,Scale=MatchLowercase}
\fi
% use upquote if available, for straight quotes in verbatim environments
\IfFileExists{upquote.sty}{\usepackage{upquote}}{}
% use microtype if available
\IfFileExists{microtype.sty}{%
\usepackage{microtype}
\UseMicrotypeSet[protrusion]{basicmath} % disable protrusion for tt fonts
}{}
\usepackage[margin=1in]{geometry}
\usepackage{hyperref}
\hypersetup{unicode=true,
            pdftitle={Mathematical Proof},
            pdfauthor={John Shea},
            pdfborder={0 0 0},
            breaklinks=true}
\urlstyle{same}  % don't use monospace font for urls
\usepackage{graphicx,grffile}
\makeatletter
\def\maxwidth{\ifdim\Gin@nat@width>\linewidth\linewidth\else\Gin@nat@width\fi}
\def\maxheight{\ifdim\Gin@nat@height>\textheight\textheight\else\Gin@nat@height\fi}
\makeatother
% Scale images if necessary, so that they will not overflow the page
% margins by default, and it is still possible to overwrite the defaults
% using explicit options in \includegraphics[width, height, ...]{}
\setkeys{Gin}{width=\maxwidth,height=\maxheight,keepaspectratio}
\IfFileExists{parskip.sty}{%
\usepackage{parskip}
}{% else
\setlength{\parindent}{0pt}
\setlength{\parskip}{6pt plus 2pt minus 1pt}
}
\setlength{\emergencystretch}{3em}  % prevent overfull lines
\providecommand{\tightlist}{%
  \setlength{\itemsep}{0pt}\setlength{\parskip}{0pt}}
\setcounter{secnumdepth}{0}
% Redefines (sub)paragraphs to behave more like sections
\ifx\paragraph\undefined\else
\let\oldparagraph\paragraph
\renewcommand{\paragraph}[1]{\oldparagraph{#1}\mbox{}}
\fi
\ifx\subparagraph\undefined\else
\let\oldsubparagraph\subparagraph
\renewcommand{\subparagraph}[1]{\oldsubparagraph{#1}\mbox{}}
\fi

%%% Use protect on footnotes to avoid problems with footnotes in titles
\let\rmarkdownfootnote\footnote%
\def\footnote{\protect\rmarkdownfootnote}

%%% Change title format to be more compact
\usepackage{titling}

% Create subtitle command for use in maketitle
\newcommand{\subtitle}[1]{
  \posttitle{
    \begin{center}\large#1\end{center}
    }
}

\setlength{\droptitle}{-2em}

  \title{Mathematical Proof}
    \pretitle{\vspace{\droptitle}\centering\huge}
  \posttitle{\par}
    \author{John Shea}
    \preauthor{\centering\large\emph}
  \postauthor{\par}
      \predate{\centering\large\emph}
  \postdate{\par}
    \date{February 03, 2019}

\usepackage{makecell}
\usepackage{booktabs}
\usepackage{float}
\usepackage{amsmath}
\usepackage{amsthm}
\usepackage{mathrsfs}
\usepackage{longtable}
\usepackage{array}
\usepackage{multirow}
\usepackage{wrapfig}
\usepackage{colortbl}
\usepackage{pdflscape}
\usepackage{tabu}
\usepackage{threeparttable}
\usepackage{threeparttablex}
\usepackage[normalem]{ulem}
\usepackage{xcolor}

\begin{document}
\maketitle

\section{Assignment \#5}\label{assignment-5}

\subsection{Question 1}\label{question-1}


\newtheorem{theorem}{Theorem}

\begin{theorem}
  If $\mathscr{F}$ and $\mathscr{G}$ are families of sets, and $A$ in
  $\mathscr{F}$ and $B$ in $\mathscr{G}$, then $\cup\mathscr{F}$ and
  $\cup\mathscr{G}$ are not disjoin if $A$ and $B$ are not disjoint.
\end{theorem}

\begin{proof}
  Suppose $A$ and $B$ are not disjoint. Then there exists at least one $x$
  which exists in $A$ and $B$. Since $A$ is an element of $\mathscr{F}$ and
  $\cup\mathscr{F}$ consists of every member of the elements of $\mathscr{F}$,
  then $x$ exists in $\cup\mathscr{F}$. And since $B$ is an element of
  $\mathscr{G}$ and $\cup\mathscr{G}$ consists of every member of every
  element of $\mathscr{G}$, then $x$ exists in $\cup\mathscr{G}$. Therefore,
  $\cup\mathscr{F}$ and $\cup\mathscr{G}$ have $x$ in common and are
  subsequently not disjoint.
\end{proof}


\subsection{Question 2}\label{question-2}


\begin{theorem}
\end{theorem}

\begin{proof}
\end{proof}

\begin{enumerate}[label=\alph*]
  \item $S^{-1} \circ R$

    $S$ bleh bleh

  \item $R^{-1} \circ S$

    $R$ bleh
\end{enumerate}


\subsection{Question 3}\label{question-3}


\begin{theorem}
  There is a unique real number $x$ such that for every real number $y$, $xy+x-17=17y$
\end{theorem}

\begin{proof}
  First, take $xy+x-17=17y$ and add $17$ to both sides, the result is
  $xy+x=17y+17$. Then factor $x+1$ out of both sides and get $x(y+1)=17(y+1)$.
  Then dived both sides by $y+1$ and get $x=17$. This proves that $x=17$ for all
  real values of $y$ except $-1$. Because, if $y=-1$ then dividing by $y+1$
  constitutes dividing by zero which is undefined. To prove that $x=17$ holds as
  true for $y=-1$, take the point where the division by zero would occur and
  insert $x=17$ and $y=-1$ to test for truth. That results in the statement
  $17(-1+1)=17(-1+1)$, which is clearly identical and leads to the true
  statement that $0=0$.
\end{proof}

Visual proof that $x=17$ for all real numbers $y \ne -1$.

\begin{align*}
  zy+x-17=17y\\
  &\equiv xy+x-17+17=17y+17 &(\textrm{add 17 to both sides})\\
  &\equiv xy+x=17y+17 &(\textrm{simplify})\\
  &\equiv x(y+1)=17(y+1) &(\textrm{factor both sides})\\
  &\equiv \frac{x(y+1)}{y+1}{y+1}=\frac{17(y+1)}{y+1} &(\textrm{divide both
  sides by y-1})\\
  &\equiv x=17 &(\textrm{conclusion})
\end{align*}


Visual proof that $x=17$ for $y = -1$.

\begin{align*}
  zy+x-17=17y\\
  &\equiv xy+x-17+17=17y+17 &(\textrm{add 17 to both sides})\\
  &\equiv xy+x=17y+17 &(\textrm{simplify})\\
  &\equiv x(y+1)=17(y+1) &(\textrm{factor both sides})\\
  &\equiv 17(-1+1)=17(-1+1) &(\textrm{insert $x=17$ and $y=-1$})\\
  &\equiv 17(0)=17(0) &(\textrm{simplify})\\
  &\equiv 0=0 &(\textrm{true statement})
\end{align*}


\subsection{Question 4}\label{question-4}


\begin{theorem}
  If $x$ is a negative real number and $x < \frac{1}{x}$, then $x < -1$.
\end{theorem}

\begin{proof}
  Suppose $a$ is the absolute value of $x$. Then the given statement can be
  written as $-a < - \frac{1}{a}$. Then multiply both sides by $-a$ and we get
  $a^2 > 1$. We then take the principle square root of both sides (because $a$
  is known to be an absolute value, it is not necessary to take the plus or
  minus square root) and arrive at $a>1$. Then we multiply both sides by $-1$
  and get $-a<-1$. Finally, we insert $x$ for $-a$ and we get the desired $x <
  -1$.
\end{proof}

A more visual depiction:

\begin{align*}
  x < 0 \land x < \frac{1}{x}\\
  &a = |x| &(\textrm{stipulation})\\
  &-a < -\frac{1}{a} &(\textrm{restatement})\\
  &-a \cdot -a < -\frac{1}{a} \cdot -a &(\textrm{mutliply both sides by -a})\\
  &a^2 > 1 &(\textrm{result of multiplication})\\
  &\sqrt{a^2} > \sqrt{1} &(\textrm{take principal square root of both sides})\\
  &a > 1 &(\textrm{result})\\
  &a \cdot -1 < 1 \cdot -1 &(\textrm{mutliply both sides by -1})\\
  &-a < -1 &(\textrm{result})\\
  &x < -1 &(\textrm{substitute x for -1})
\end{align*}


\subsection{Question 5}\label{question-5}


\begin{theorem}
  For every positive integer $n$, there is a sequence of $2n$ consecutive
  positive integers containing no primes.
\end{theorem}

\begin{proof}
  Suppose $n$ is a positive integer. Suppose $x=(2n+1)!+2$. $x= 1 \cdot 2 \cdot
  3 \cdot 4 ...(2n+1) + 2)$. Two can be factored out to create $2 \cdot (1 \cdot
  3 \cdot 4 . . . (2n+1) + 1)$. If $2$ can be factored out of $x$, then $x$ is
  divisible by $2$ and therefore not prime. Similarly, $x+1=1 \cdot 2 \cdot 3
  \cdot 4 ...(2n+1)+3$. Three can be factored out leaving $3 \cdot (1 \cdot 2
  \cdot 4 . . . (2n+1) +1)$. With $3$ factorizable, $x+1$ is divisible by $3$ and
  therefore not prime. This pattern repeats for all the numbers in the sequence
  proving that for any positive integer $n$, There is a sequence of $2n$
  consecutive positive integers containing no primes.
\end{proof}

A more visual depiction:

\begin{align*}
      x &= 1 \cdot 2 \cdot 3 \cdot 4 . . . (2n+1) + 2\\
        &= 2 \cdot (1 \cdot 3 \cdot 4 . . . (2n+1) + 1) &(\textrm{$x$ not prime})\\
  x + 1 &= 1 \cdot 2 \cdot 3 \cdot 4 . . . (2n+1) +3\\
        &= 3 \cdot (1 \cdot 2 \cdot 4 . . . (2n+1) +1) &(\textrm{$x+1$ not prime})\\
  x + 2 &= 1 \cdot 2 \cdot 3 \cdot 4 . . . (2n+1) +4\\
        &= 4 \cdot (1 \cdot 2 \cdot 3 . . . (2n+1) +1) &(\textrm{$x+2$ not prime})\\
        &... &(\textrm{pattern repeats for the remainder of the sequence})
\end{align*}


\subsection{Question 6}\label{question-6}


\begin{theorem}
  For $f(x)=x^2$ with domain $0 \le x \le 10$ the $\lim_{x \to 5}f(x)=25$
\end{theorem}

\begin{proof}
\end{proof}



\end{document}
