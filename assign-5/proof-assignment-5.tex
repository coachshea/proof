\documentclass[12pt,]{article}
\usepackage{lmodern}
\usepackage{amssymb,amsmath}
\usepackage{ifxetex,ifluatex}
\usepackage{fixltx2e} % provides \textsubscript
\ifnum 0\ifxetex 1\fi\ifluatex 1\fi=0 % if pdftex
  \usepackage[T1]{fontenc}
  \usepackage[utf8]{inputenc}
\else % if luatex or xelatex
  \ifxetex
    \usepackage{mathspec}
  \else
    \usepackage{fontspec}
  \fi
  \defaultfontfeatures{Ligatures=TeX,Scale=MatchLowercase}
\fi
% use upquote if available, for straight quotes in verbatim environments
\IfFileExists{upquote.sty}{\usepackage{upquote}}{}
% use microtype if available
\IfFileExists{microtype.sty}{%
\usepackage{microtype}
\UseMicrotypeSet[protrusion]{basicmath} % disable protrusion for tt fonts
}{}
\usepackage[margin=1in]{geometry}
\usepackage{hyperref}
\hypersetup{unicode=true,
            pdftitle={Mathematical Proof},
            pdfauthor={John Shea},
            pdfborder={0 0 0},
            breaklinks=true}
\urlstyle{same}  % don't use monospace font for urls
\usepackage{graphicx,grffile}
\makeatletter
\def\maxwidth{\ifdim\Gin@nat@width>\linewidth\linewidth\else\Gin@nat@width\fi}
\def\maxheight{\ifdim\Gin@nat@height>\textheight\textheight\else\Gin@nat@height\fi}
\makeatother
% Scale images if necessary, so that they will not overflow the page
% margins by default, and it is still possible to overwrite the defaults
% using explicit options in \includegraphics[width, height, ...]{}
\setkeys{Gin}{width=\maxwidth,height=\maxheight,keepaspectratio}
\IfFileExists{parskip.sty}{%
\usepackage{parskip}
}{% else
\setlength{\parindent}{0pt}
\setlength{\parskip}{6pt plus 2pt minus 1pt}
}
\setlength{\emergencystretch}{3em}  % prevent overfull lines
\providecommand{\tightlist}{%
  \setlength{\itemsep}{0pt}\setlength{\parskip}{0pt}}
\setcounter{secnumdepth}{0}
% Redefines (sub)paragraphs to behave more like sections
\ifx\paragraph\undefined\else
\let\oldparagraph\paragraph
\renewcommand{\paragraph}[1]{\oldparagraph{#1}\mbox{}}
\fi
\ifx\subparagraph\undefined\else
\let\oldsubparagraph\subparagraph
\renewcommand{\subparagraph}[1]{\oldsubparagraph{#1}\mbox{}}
\fi

%%% Use protect on footnotes to avoid problems with footnotes in titles
\let\rmarkdownfootnote\footnote%
\def\footnote{\protect\rmarkdownfootnote}

%%% Change title format to be more compact
\usepackage{titling}

% Create subtitle command for use in maketitle
\newcommand{\subtitle}[1]{
  \posttitle{
    \begin{center}\large#1\end{center}
    }
}

\setlength{\droptitle}{-2em}

  \title{Mathematical Proof}
    \pretitle{\vspace{\droptitle}\centering\huge}
  \posttitle{\par}
    \author{John Shea}
    \preauthor{\centering\large\emph}
  \postauthor{\par}
      \predate{\centering\large\emph}
  \postdate{\par}
    \date{February 03, 2019}

\usepackage{makecell}
\usepackage{booktabs}
\usepackage{float}
\usepackage{amsmath}
\usepackage{amsthm}
\usepackage{mathrsfs}
\usepackage{longtable}
\usepackage{array}
\usepackage{multirow}
\usepackage{wrapfig}
\usepackage{colortbl}
\usepackage{pdflscape}
\usepackage{tabu}
\usepackage{threeparttable}
\usepackage{threeparttablex}
\usepackage[normalem]{ulem}
\usepackage{xcolor}

\begin{document}
\maketitle

\section{Assignment \#5}\label{assignment-5}

\subsection{Question 1}\label{question-1}



\newtheorem{theorem}{Theorem}

\begin{theorem}
  If $A$ is a set and $\{B_i | i \in I\}$ is an indexed family of sets. $A
  \times (\cup_{i\in I}B_i) =\cup_{i\in I})(A \times B)$.
\end{theorem}

\begin{proof}
  Suppose $(a,b) \in A \times (\cup_{i\in I}B_i)$. Then $a \in A$ and $b \in
  \cup_{i\in I}$.
  so $A \times (\cup_{i\in I}B_i) \subseteq \cup_{i\in I})(A \times B)$.

  Suppose $(x,y) \in \cup_{i\in I}(A \times B)$. Then $x \in$ So, $\cup_{i\in
  I})(A \times B) \subseteq A \times (\cup_{i\in I}B_i)$.
  Therefore $A \times (\cup_{i\in I}=\cup_{i\in I})(A \times B)$.
\end{proof}


\subsection{Question 2}\label{question-2}


\begin{theorem}
  If $A \subseteq B$ and $x \in A$, but $x \notin B\setminus C$, then $x
  \in C$. 
\end{theorem}

\begin{proof}
  Suppose $x \notin C$. Then
\end{proof}



\begin{align*}
\end{align*}


\subsection{Question 3}\label{question-3}


\begin{theorem}
  If $A\setminus B \subseteq C \cap D$ and $x \in A$, but $x \notin C$, then $x
  \in B$
\end{theorem}

\begin{proof}
  Suppose
\end{proof}



\begin{align*}
\end{align*}


\subsection{Question 4}\label{question-4}


\begin{theorem}
  For any set $U$, for every $B\in \wp(U)$ there is a unique $D$ such that for
  every $C \in \wp(U), C \setminus B = C \cap D$.
\end{theorem}

\begin{proof}
  Because $B$, $D$, and $C$ are within $\wp(U)$, $\wp(U)$ is the universal set
  within the context of this problem. Suppose $D$ is the complement of $B$
  (i.e. $B^c$) within set $U$. Then $D$ is the set of all $x \notin B$ within
  $\wp(U)$. Therefore, for any possible $C \in \wp(U)$ if the members of $C$
  which also exists in $B$ are removed (i.e. $C \setminus B$), the remainder
  of the set will exist in both $C$ and $D$. Therefore, there exists a unique
  set $D$ such that for all $C$, $C \setminus B = C \cap D$. Furthermore, that
  unique set $D$ is the compliment of $B$, $B^c$.
\end{proof}


\subsection{Question 5}\label{question-5}


\begin{theorem}
  For every positive integer $n$, there is a sequence of $2n$ consecutive
  positive integers containing no primes.
\end{theorem}

\begin{proof}
\end{proof}


\subsection{Question 6}\label{question-6}


\begin{theorem}
  The relation $R$ on $\mathbb{Z}$, such that $(a,b)\in R$ if and only if $a$
  and $b$, when written out, have the same number of $5s$ is an equivalence
  relation.
\end{theorem}

\begin{proof}
  Suppose $\{B_i|i\in \mathbb{Z}^+\}$ is an indexed family of sets such that
  $B_1$ contains all of the infinitely many numbers which contain one $5$, and
  $B_2$ contains all of the infinitely many numbers which contain two $5s$,
  and so on for an infinitely number of sets $B_i$. Clearly, no number can
  reside in more than one of the subsets of $B$. Therefore, we can restate the
  definition of $R$ as $(a,b) \in R$ if, and only if, $a \in B_i$ and $b \in
  B_i$ for some unique $i \in \mathbb{Z}^+$

  Relation $R$ is such that if, and only if, for a given $(a,b)$, both $a$ and
  $b$ are taken from the same subset of $B$, then $(a,b) \in R$. Thus, for
  every $a$ in a given subset of $B$, $(a,a)$ would always meet the criteria
  for being an element of $R$ and therefore, $R$ is reflexive.

  Also, because for any given $(a,b) \in R$, both $a$ and $b$ reside in the same
  subset of $B$, $(b,a)$ would meet the criteria for inclusion in $R$.
  Therefore, R is symmetric.

  Lastly, if $(a,b)$ and $(b,c)$ are elements of $R$, then $a$, $b$, and $c$ are
  all elements of the same subsets of $B$. Therefore $(a,c)$ meets the
  requirement of a transitive relation.

  Taken together, $R$ is proven to be reflexive, symmetric, and transitive; and
  therefore, $R$ is an equivalence relation.
\end{proof}



\end{document}
