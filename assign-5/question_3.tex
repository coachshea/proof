
\begin{theorem}
  There is a unique real number $x$ such that for every real number $y$, $xy+x–17=17y$
\end{theorem}

\begin{proof}
  First, take $xy+x-17=17y$ and add $17$ to both sides, the result is
  $xy+x=17y+17$. Then factor $x+1$ out of both sides and get $x(y+1)=17(y+1)$.
  Then dived both sides by $y+1$ and get $x=17$. This proves that $x=17$ for all
  real values of $y$ except $-1$. Because, if $y=-1$ then dividing by $y+1$
  constitutes dividing by zero which is undefined. To prove that $x=17$ holds as
  true for $y=-1$, take the point where the division by zero would occur and
  insert $x=17$ and $y=-1$ to test for truth. That results in the statement
  $17(-1+1)=17(-1+1)$, which is clearly identical and leads to the true
  statement that $0=0$.
\end{proof}

Visual proof that $x=17$ for all real numbers except $-1$.

\begin{align*}
  zy+x-17=17y\\
  &\equiv xy+x-17+17=17y+17 &(\textrm{add 17 to both sides})\\
  &\equiv xy+x=17y+17 &(\textrm{simplify})\\
  &\equiv x(y+1)=17(y+1) &(\textrm{factor both sides})\\
  &\equiv \frac{x(y+1)}{y+1}{y+1}=\frac{17(y+1)}{y+1} &(\textrm{divide both
  sides by y-1})\\
  &\equiv x=17 &(\textrm{conclusion})
\end{align*}


Visual proof that $x=17$ for $-10$.

\begin{align*}
  zy+x-17=17y\\
  &\equiv xy+x-17+17=17y+17 &(\textrm{add 17 to both sides})\\
  &\equiv xy+x=17y+17 &(\textrm{simplify})\\
  &\equiv x(y+1)=17(y+1) &(\textrm{factor both sides})\\
  &\equiv 17(-1+1)=17(-1+1) &(\textrm{insert $x=17$ and $y=-1$})\\
  &\equiv 17(0)=17(0) &(\textrm{simplify})\\
  &\equiv 0=0 &(\textrm{true statement})
\end{align*}
