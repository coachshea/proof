The graph depicts the following relation:
$\{(a,a),(a,b),(a,c),(b,d),(d,b),(d,a)\}$.

That the relation is not reflexive is proved by the fact that there is no
$(b,b), (c,c),$ or $(d,d)$ in the relation. The only example of a reflexive
relationship is $(a,a)$. But the definition of reflexive relation requires
reflexivity for all elements of the relation.

That the relation is not symmetric is proved by the fact that there is an
$(a,c)$ yet no $(c,a)$, an $(a,b)$ and no $(b,a)$, and a $(d,a)$, but no
$(a,d)$. There is an example of a symmetric relationships $(b,d),(d,b)$. But
once again, symmetry calls for all relations to be reciprocated, and the
counter-examples clearly belie symmetry.

Finally, that the relation is not transitive is proved by the fact that
there is an $(a,b)$ and a $(b,d)$, but no $(a,d)$. There is also a $(d,a)$
and $(a,c)$, yet there is not a $(d,c)$. There is, however, a $(a,b)$ and a
$(d,b)$, but again, all pairs need to be transitive, for the relation to meet
the definition of a transitive relation.
