
\begin{theorem}
\end{theorem}

\begin{proof}
\end{proof}

\begin{enumerate}[label=\alph*]
  \item $S^{-1} \circ R$

    $S=\{(4,a),(4,d),(5,b),(5,c)\}$, and is a relation from $B$ to $C$, so
    $S^{-1}=\{(a,4),(d,4),(b,5),(c,5)\}$ and is a relation from $C$ to $B$.
    $R=\{(1,b),(2,a),(2,b),(2,c),(3,d)\}$ and is a relation from $A$ to $C$.
    Therefore $S^{-1} \circ R = \{(1,5),(2,4),(2,5)\}$ and is a relation from
    $A$ to $B$. This results because we now have a relation that begins in $A$,
    connects through $C$ and arrives at $B$. The final relation $S^{-1} \circ R$
    becomes a set of ordered pairs that have elements of $A$ as the first
    coordinate and elements of $B$ as the second.

  \item $R^{-1} \circ S$

    $R=\{(1,b),(2,a),(2,b),(2,c),(3,d)\}$ and is a relation from $A$ to $C$,
    $Rso ^{-1}=\{(b,1),(a,2),(b,2),(c,2),(d,3)$ and is a relation from $C$ to
    $A$. $S=\{(4,a),(4,d),(5,b),(5,c)\}$ and is a relation from $B$ to $C$.
    Therefore $R^{-1} \circ S$ is a relation from $B$ to $A$. This results
    because we now have a relation that begins in $B$, connects through $C$ and
    arrives at $A$. The final relation $R^{-1} \circ S$ becomes a set of ordered
    pairs that have elements of $B$ as the first coordinate and elements of $A$
    as the second.
\end{enumerate}
