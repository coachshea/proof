
\newtheorem{theorem}{Theorem}

\begin{theorem}
  If $0 < \frac{1}{a} < \frac{1}{b}$, then $b < a.$
\end{theorem}

\begin{proof}
  Suppose $0 < \frac{1}{a} < \frac{1}{b}$ Because both $\frac{1}{a}$ and
  $\frac{1}{b}$ are both greater than zero, both $a$ and $b$ must be positive.
  Therefore, all terms can be multiplied by $a$ and $b$ without needing to
  change the direction of the inequality. Thus, when all three terms are
  multiplied by $a$, the result is $0 < 1 < \frac{a}{b}$. Then when all three
  terms are multiplied by $b$, the result is $0 < b < a$. Therefore $b$ is less
  than $a$.
\end{proof}

A more visual depiction:


\begin{align*}
  0 < \frac{1}{a} < \frac{1}{b}\\  
  &\equiv 0 \cdot a < \frac{1}{a} \cdot a < \frac{1}{b} \cdot a
    &(\textrm{multiply all terms by a})\\
  &\equiv 0 < 1 < \frac{a}{b} &(\textrm{simplify})\\
  &\equiv 0 \cdot b < 1 \cdot b < \frac{a}{b} \cdot b &(\textrm{multiply all
    terms by b})\\
  & \equiv 0 < b < a &(\textrm{simplify})\\
  & b < a &(\textrm{conclusion})
\end{align*}
