
\begin{theorem}
  If $x$ is a negative real number and $x < \frac{1}{x}$, then $x < -1$.
\end{theorem}

\begin{proof}
  Suppose $a$ is the absolute value of $x$. Then the given statement can be
  written as $-a < - \frac{1}{a}$. Then multiply both sides by $-a$ and we get
  $a^2 > 1$. We then take the principle square root of both sides (because $a$
  is known to be an absolute value, it is not necessary to take the plus or
  minus square root) and arrive at $a>1$. Then we multiply both sides by $-1$
  and get $-a<-1$. Finally, we insert $x$ for $-a$ and we get the desired $x <
  -1$.
\end{proof}

A more visual depiction:

\begin{align*}
  x < 0 \land x < \frac{1}{x}\\
  &a = |x| &(\textrm{stipulation})\\
  &-a < -\frac{1}{a} &(\textrm{restatement})\\
  &-a \cdot -a < -\frac{1}{a} \cdot -a &(\textrm{mutliply both sides by -a})\\
  &a^2 > 1 &(\textrm{result of multiplication})\\
  &\sqrt{a^2} > \sqrt{1} &(\textrm{take principal square root of both sides})\\
  &a > 1 &(\textrm{result})\\
  &a \cdot -1 < 1 \cdot -1 &(\textrm{mutliply both sides by -1})\\
  &-a < -1 &(\textrm{result})\\
  &x < -1 &(\textrm{substitute x for -1})
\end{align*}
