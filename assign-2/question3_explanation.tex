$A \cup B$ is similar to $C \cup (A \setminus B)$, with the exception of any
element that is in A and B, but not C (i.e. $(A \cap B) \setminus C$). For
example: if we have sets $A=\{1, 4, 5, 7\}$, $B=\{2, 4, 6, 7\}$, and $C=\{3,
5, 6, 7\}$. $A \cup C = \{1, 3, 4, 5, 6, 7\}$. Whereas $C \cup (A \setminus
B) = \{1, 3, 5, 6, 7\}$. This is created by taking $A \setminus B = \{1, 5\}$
and adding $C$. The difference between the two sets is $A \cap B \setminus
C=\{4\}$. All of which is depicted below.


\begin{minipage}[t]{0.4\linewidth}
\begin{center}
    $A \cup C$\\
\begin{venndiagram3sets}[labelOnlyA={1},
                         labelOnlyB={2},
                         labelOnlyC={3},
                         labelOnlyAB={4},
                         labelOnlyAC={5},
                         labelOnlyBC={6},
                         labelABC={7},labelNotABC={8}]
    \fillA \fillC
\end{venndiagram3sets}
\end{center}
\end{minipage}
\begin{minipage}[t]{0.4\linewidth}
\begin{center}
    $C \cup (A \setminus B)$\\
\begin{venndiagram3sets}[labelOnlyA={1},
                         labelOnlyB={2},
                         labelOnlyC={3},
                         labelOnlyAB={4},
                         labelOnlyAC={5},
                         labelOnlyBC={6},
                         labelABC={7},labelNotABC={8}]
    \fillANotB \fillC
\end{venndiagram3sets}
\end{center}
\end{minipage}

\begin{center}
    $A \cap B \setminus C$\\
\begin{venndiagram3sets}[labelOnlyA={1},
                         labelOnlyB={2},
                         labelOnlyC={3},
                         labelOnlyAB={4},
                         labelOnlyAC={5},
                         labelOnlyBC={6},
                         labelABC={7},labelNotABC={8}]
    \fillACapBNotC
\end{venndiagram3sets}
\end{center}
